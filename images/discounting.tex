\documentclass{article}
\usepackage{lipsum}
\usepackage{setspace}
\usepackage{nicematrix}
 \usepackage{relsize}
\NiceMatrixOptions{cell-space-limits=1pt}
 \onehalfspacing
\linespread{2}
\begin{document}


\begin{itemize}
  \item \textbf{Nature of Cash Flows:}
  \begin{itemize}
    \item \textbf{Discrete Time:} Uses \( \frac{FV}{(1 + r)^t} \) for a single, lump-sum future payment or a series of discrete payments at specific time points.
    \item \textbf{Continuous Time with \( \exp(-rt) \):} Used for discounting a single, lump-sum future payment or a series of discrete payments at specific time points.
    \item \textbf{Continuous Time with \( \frac{1}{r}(1 - \exp(-rt)) \):} Used for discounting a continuous payment stream, like a continuously paid annuity.
  \end{itemize}

  \item \textbf{Formula Interpretation:}
  \begin{itemize}
    \item \textbf{Discrete Time:} \( \frac{FV}{(1 + r)^t} \) directly gives the present value of a future cash flow at discrete time intervals.
    \item \textbf{Continuous Time with \( \exp(-rt) \):} Directly gives the present value of a future cash flow. It's straightforward and simple for single cash flows at discrete times.
    \item \textbf{Continuous Time with \( \frac{1}{r}(1 - \exp(-rt)) \):} Gives the present value of the entire continuous cash flow from time 0 to \( t \). It's more complex and not directly applicable for discrete time points or single cash flows.
  \end{itemize}

  \item \textbf{Simplicity and Conventional Use:}
  \begin{itemize}
    \item \textbf{Discrete Time:} \( \frac{FV}{(1 + r)^t} \) is widely accepted and used for its simplicity and direct interpretation for discounting future values at discrete intervals.
    \item \textbf{Continuous Time with \( \exp(-rt) \):} Widely accepted and used due to its simplicity and direct interpretation for discounting future values.
    \item \textbf{Continuous Time with \( \frac{1}{r}(1 - \exp(-rt)) \):} More specialized and less commonly applied outside its intended context of continuous cash flows due to its complexity for single or discrete cash flows.
  \end{itemize}
\end{itemize}
\end{document}